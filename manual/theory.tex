\chapter{Theory}

In this section we detail the theoretical background of the \bdistrib~code.
Our discussion closely follows Ref \cite{nescoil}
and sections 2.13 - 2.3 of Ref \cite{boozer2015}.

\section{Current potential}

Consider a surface defined by $\vect{r}(u,v)$, i.e. the position vector $\vect{r}$ along the surface
is parametrized by
two coordinates $u$ and $v$.  A vector normal to the surface is
\begin{equation}
\vect{N} = \frac{\partial \vect{r}}{\partial v} \times \frac{\partial\vect{r}}{\partial u}.
\end{equation}
(We will follow the same sign convention as \cite{nescoil}.)
This vector does not generally have unit magnitude; a unit normal vector is
\begin{equation}
\vect{n} = \frac{\vect{N}}{|\vect{n}|}.
\end{equation}

A divergence-free surface current on the surface must have the form
\begin{equation}
\vect{K} = \vect{n}\times\nabla \Phi
\label{eq:K}
\end{equation}
for a ``current potential'' $\Phi$. If we now specialize to the case of a toroidal surface,
$\Phi$ may be not be singly valued if there is net poloidal or toroidal current on the surface.
Here we restrict our attention to the single-valued part of $\Phi$.

\section{Magnetic dipole formula}

Let us now examine the magnetic field produced by the surface current (\ref{eq:K}).
We begin with the Biot-Savart law for a surface current:
\begin{equation}
\vect{B}(\vect{r}) = \frac{\mu_0}{4\pi} \int d^2r' \frac{\vect{K}(\vect{r}')\times(\vect{r}-\vect{r}')}
{|\vect{r}-\vect{r}'|^3}.
\label{eq:biot}
\end{equation}

We specialize to the case in which $u \in [0,1)$ is the poloidal coordinate, and the toroidal coordinate
$v$ increases from 0 to 1 in a single toroidal period, where there are $n_p$ identical toroidal periods.
(For instance, $n_p=5$ for W7-X.) In both \vmec~and \bdistrib, $n_p$ is called {\ttfamily nfp}.
Then any area integral on the toroidal surface (such as the integral in (\ref{eq:biot})) can be expressed as
\begin{equation}
\int d^2r\;Q = \sum_{\ell=1}^{n_p} \int_0^1 du \int_0^1 dv 
\left| \frac{\partial \vect{r}}{\partial u}\times\frac{\partial\vect{r}}{\partial v}\right|Q
=\sum_{\ell=1}^{n_p} \int_0^1 du \int_0^1 dv |\vect{N}|Q
\end{equation}
for any quantity $Q$.
Thus, (\ref{eq:biot})
becomes
\begin{equation}
\vect{B}(\vect{r})
=\frac{\mu_0}{4\pi}
\sum_{\ell'=1}^{n_p} \int_0^1 du' \int_0^1 dv' \frac{1}{|\vect{r}-\vect{r}'|}
\left[ \vect{N}'\times\nabla' \Phi'\right]\times(\vect{r}-\vect{r}').
\end{equation}
We use primes to denote quantities on the surface with the current potential,
to distinguish these quantities from the coordinates at which $\vect{B}$ is evaluated.

Now observe that $\nabla'\Phi' = (\nabla' u')\partial\Phi'/\partial u' + (\nabla' v')\partial\Phi'/\partial v'$.
Also observe
(suppressing primes temporarily to reduce clutter)
\begin{equation}
\vect{N}\times\nabla u
=\left(\frac{\partial\vect{r}}{\partial v}\times\frac{\partial \vect{r}}{\partial u}\right)
\times\nabla u
=
\left(\frac{\partial\vect{r}}{\partial u}\right)
\underbrace{\left(\frac{\partial\vect{r}}{\partial v}\cdot\nabla u\right)}_{=0}
-
\left(\frac{\partial\vect{r}}{\partial v}\right)
\underbrace{\left(\frac{\partial\vect{r}}{\partial u}\cdot\nabla u\right)}_{=1}
=
-\frac{\partial\vect{r}}{\partial v}
\end{equation}
and
\begin{equation}
\vect{N}\times\nabla v
=\left(\frac{\partial\vect{r}}{\partial v}\times\frac{\partial \vect{r}}{\partial u}\right)
\times\nabla v
=
\left(\frac{\partial\vect{r}}{\partial u}\right)
\underbrace{\left(\frac{\partial\vect{r}}{\partial v}\cdot\nabla v\right)}_{=1}
-
\left(\frac{\partial\vect{r}}{\partial v}\right)
\underbrace{\left(\frac{\partial\vect{r}}{\partial u}\cdot\nabla v\right)}_{=0}
=
\frac{\partial\vect{r}}{\partial u}.
\end{equation}
Thus, we obtain
\begin{equation}
\vect{B}(\vect{r}) = \frac{\mu_0}{4\pi}
\sum_{\ell'=1}^{n_p} \int_0^1 du' \int_0^1 dv' \frac{1}{|\vect{r}-\vect{r}'|^3}
\left[ \frac{\partial\vect{r}'}{\partial u'} \frac{\partial\Phi'}{\partial v'}
- \frac{\partial\vect{r}'}{\partial v'} \frac{\partial\Phi'}{\partial u'} \right]
\times(\vect{r}-\vect{r}').
\end{equation}
Integrating by parts in $u'$ and $v'$,
and using the assumption that $\Phi$ is single-valued to drop boundary contributions,
\begin{eqnarray}
\vect{B}(\vect{r}) 
&=& \frac{\mu_0}{4\pi}
\sum_{\ell'=1}^{n_p} \int_0^1 du' \int_0^1 dv' \; \Phi'
\left[
\frac{\partial}{\partial u'}
\left(  \frac{1}{|\vect{r}-\vect{r}'|^3}
\frac{\partial \vect{r}'}{\partial v'}\times (\vect{r}-\vect{r}')
\right) \right. \\
&&
\hspace{1.6in}-
\left.
\frac{\partial}{\partial v'}
\left( \frac{1}{|\vect{r}-\vect{r}'|^3}
\frac{\partial \vect{r}'}{\partial u'}\times (\vect{r}-\vect{r}')
\right)
\right]. \nonumber
\end{eqnarray}
If the outer derivatives act on the $\partial\vect{r}'/\partial v'$
and $\partial\vect{r}'/\partial u'$ terms, the resulting contributions cancel,
so we can pull these $\partial\vect{r}'/\partial \ldots$ terms out in front of the outer
derivatives:
\begin{equation}
\vect{B}(\vect{r}) = \frac{\mu_0}{4\pi}
\sum_{\ell'=1}^{n_p} \int_0^1 du' \int_0^1 dv' \; \Phi'
\left[
\frac{\partial\vect{r}'}{\partial v'}
\times
\frac{\partial}{\partial u'}
\left(  \frac{\vect{r}-\vect{r}'}{|\vect{r}-\vect{r}'|^3}
\right)
-
\frac{\partial\vect{r}'}{\partial u'}
\times
\frac{\partial}{\partial v'}
\left( \frac{\vect{r}-\vect{r}'}{|\vect{r}-\vect{r}'|^3}
\right)
\right].
\label{eq:13}
\end{equation}

Next, note that for any scalar $w$ and vector $\vect{a}(w)$,
\begin{equation}
\frac{\partial}{\partial w}
\left(\frac{\vect{a}}{|\vect{a}|^3}\right)
=
\frac{1}{|\vect{a}|^3}
\frac{\partial\vect{a}}{\partial w}
-
\frac{3}{|\vect{a}|^5}
\vect{a}\vect{a}\cdot
\frac{\partial\vect{a}}{\partial w}.
\end{equation}
Using this result, and multiplying (\ref{eq:13}) by an arbitrary vector $\vect{Y}$,
\begin{eqnarray}
\vect{B}\cdot\vect{Y}
= \frac{\mu_0}{4\pi}
\sum_{\ell'=1}^{n_p} \int_0^1 du' \int_0^1 dv' \; \Phi'
\left[
...
\right]
\end{eqnarray}

\section{Basis functions}

\section{Fourier basis}

\section{Transfer matrix}

\section{Cylindrical approximation}



\label{sec:requirements}

