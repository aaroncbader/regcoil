\chapter{Input Parameters}
\label{ch:input}

\newcommand{\param}[5]{{\setlength{\parindent}{0cm} {\ttfamily \bfseries \hypertarget{#1}{#1}}\\{\it Type}: #2\\{\it Default}: #3\\{\it When it matters}: #4\\{\it Meaning}: #5}}
\newcommand{\myhrule}{{\setlength{\parindent}{0cm} \hrulefill }}

\newcommand{\true}{{\ttfamily .true.}}
\newcommand{\false}{{\ttfamily .false.}}

In this section we describe all the parameters which can be included in the input namelist. 

\section{General parameters}

\param{general\_option}
{integer}
{1}
{Always}
{Determines the overall flow of program execution.\\

{\ttfamily general\_option} = 1: Compute the current potential for a range of lambda.\\

{\ttfamily general\_option} = 2: Do not compute the current potential, but rather load the current potential computed
by NESCOIL in the file \parlink{nescout\_filename}, reporting the $\chi^2_B$ and $\chi^2_K$ for it.
}

\myhrule

\param{symmetry\_option}
{integer}
{1}
{Always}
{Determines which set of basis functions is used.\\

{\ttfamily symmetry\_option} = 1: Use $\sin(2 \pi [mu+nv])$ basis functions.\\

{\ttfamily symmetry\_option} = 2: Use $\cos(2 \pi [mu+nv])$ basis functions.\\

{\ttfamily symmetry\_option} = 3: Use both $\sin(2 \pi [mu+nv])$ and $\cos(2 \pi [mu+nv])$ basis functions.
}

\myhrule

\param{basis\_option\_plasma}
{integer}
{1}
{Always}
{Determines which basis functions are used, as well as which weight $w$ is used to define orthogonality.
This option controls only the plasma surface. Analogous options {\ttfamily basis\_option\_middle} and {\ttfamily basis\_option\_outer}
exist for the other two surfaces.\\

{\ttfamily basis\_option\_plasma} = 1: Use $w = 1 / (n_p |\vect{N}|)$, so the basis functions
are the Fourier functions $\sqrt{2} \sin(2\pi[mu+nv])$ and/or $\sqrt{2} \cos(2\pi[mu+nv])$ (
as determined by \parlink{symmetry\_option}).\\

{\ttfamily basis\_option\_plasma} = 2: Use a constant weight, $w = 1/A$ where $A$ is the surface area. The basis functions
will be Fourier functions scaled by $\sqrt{2A/[n_p |\vect{N}|]}$.\\

{\ttfamily basis\_option\_plasma} = 3: Use a constant weight, $w = 1/A$. The basis functions
will be linear combinations of the Fourier functions, as constructed using the Cholesky method.
}

\myhrule

\param{n\_singular\_vectors\_to\_save}
{integer}
{12}
{Always}
{Number of columns of $U$ and $V$ to save in the output file, where $U$ and $V$ are the matrices of left and right singular vectors of the transfer matrix,
$T = U\Sigma V^{T}$.
Regardless of this parameter, all singular \emph{values} will be saved in the output file.}

\myhrule

\param{save\_level}
{integer}
{2}
{Always}
{Option related determining how many variables are saved in the \netCDF~output file.  The larger the value, the smaller the output file.\\

{\ttfamily save\_level} = 0: Save everything.\\

{\ttfamily save\_level} = 1: Do not save the inductance matrix.\\

{\ttfamily save\_level} = 2: Also do not save the matrix $g$.\\
}

\myhrule

\param{net\_poloidal\_current\_Amperes}
{real}
{1.0}
{If \parlink{geometry\_option\_plasma}=0,1,or 5, i.e. if the plasma surface is not a vmec equilibrium.}
{The number of Amperes of current the links the coil winding surface poloidally,
denoted $G$ in \cite{regcoilPaper}. If the plasma surface is obtained from a vmec equilibrium,
then {\ttfamily net\_poloidal\_current\_Amperes} will be determined instead
from the {\ttfamily bvco} value in the vmec wout file.
}

\myhrule

\param{net\_toroidal\_current\_Amperes}
{real}
{0.0}
{Always}
{The number of Amperes of current the links the coil winding surface toroidally,
denoted $I$ in \cite{regcoilPaper}. Unlike the net poloidal current, this number
is never read from a wout file.
}



\section{Resolution parameters}

For any new set of surface geometries you consider, you should vary the resolution parameters in this section to make sure
they are large enough.  These parameters should be large enough that the code results you care about are unchanged under further
resolution increases.

\myhrule

\param{ntheta\_plasma}
{integer}
{64}
{Always}
{Number of grid points in poloidal angle used to evaluate surface integrals on the plasma surface.
Often 64 or 128 is a good value.
It is resonable and common but not mandatory to use the same value for {\ttfamily ntheta\_plasma} and \parlink{ntheta\_coil}.}

\myhrule

\param{ntheta\_coil}
{integer}
{64}
{Always}
{Number of grid points in poloidal angle used to evaluate surface integrals on the coil winding surface.
Often 64 or 128 is a good value.
It is resonable and common but not mandatory to use the same value for \parlink{ntheta\_plasma} and {\ttfamily ntheta\_coil}.}

\myhrule


\param{nzeta\_plasma}
{integer}
{64}
{Always}
{Number of grid points in toroidal angle used to evaluate surface integrals on the plasma surface.
Often 64 or 128 is a good value.
It is resonable and common but not mandatory to use the same value for {\ttfamily nzeta\_plasma} and \parlink{nzeta\_coil}.}

\myhrule

\param{nzeta\_coil}
{integer}
{64}
{Always}
{Number of grid points in toroidal angle used to evaluate surface integrals on the coil winding surface.
Often 64 or 128 is a good value.
It is resonable and common but not mandatory to use the same value for \parlink{nzeta\_plasma} and {\ttfamily nzeta\_coil}.}

\myhrule

\param{mpol\_coil}
{integer}
{8}
{Always}
{Maximum poloidal mode number to include for the single-valued part of the current potential on the coil winding surface.
}

\myhrule

\param{ntor\_coil}
{integer}
{8}
{Always}
{
Maximum toroidal mode number to include for the single-valued part of the current potential on the coil winding surface.
}

\myhrule

\param{mpol\_transform\_refinement}
{real}
{5.0}
{Only when \parlink{geometry\_option\_plasma} is 4.}
{The number of poloidal mode numbers in the \vmec~file will be multiplied by this value
when transforming from the original poloidal angle to the straight-field-line angle.
Since the original \vmec~angle is chosen to minimize the number of Fourier modes required,
more modes are required in any other coordinate.
This parameter affects the time required to compute constant-offset surfaces,
but does not affect the time for other calculations.
}

\myhrule

\param{ntor\_transform\_refinement}
{real}
{1.0}
{Only when \parlink{geometry\_option\_plasma} is 4.}
{The number of toroidal mode numbers in the \vmec~file will be multiplied by this value
when transforming from the original poloidal angle to the straight-field-line angle.
Since the original \vmec~angle is chosen to minimize the number of Fourier modes required,
more modes are required in any other coordinate.
This parameter affects the time required to compute constant-offset surfaces,
but does not affect the time for other calculations.
}

\section{Geometry parameters for the plasma surface}

\param{geometry\_option\_plasma}
{integer}
{0}
{Always}
{This option controls which type of geometry is used for the plasma surface.\\

{\ttfamily geometry\_option\_plasma} = 0: The plasma surface will be a plain circular torus. The major radius will be \parlink{R0\_plasma}.
     The minor radius will be \parlink{a\_plasma}. This option exists just for testing purposes.\\

{\ttfamily geometry\_option\_plasma} = 1: Identical to option 0.\\

{\ttfamily geometry\_option\_plasma} = 2: The plasma surface will be the last surface in the full radial grid of the \vmec~file specified by \parlink{wout\_ilename}.
The poloidal angle used will be the normal \vmec~angle which is not a straight-field-line coordinate.
This is typically the best option to use for working with \vmec~equilibria.\\

{\ttfamily geometry\_option\_plasma} = 3: The plasma surface will be the last surface in the half radial grid of the \vmec~file specified by \parlink{wout\_filename}.
The poloidal angle used will be the normal \vmec~angle which is not a straight-field-line coordinate.
This option exists so that the same flux surface can be used when comparing with {\ttfamily geometry\_option\_plasma} = 4.\\

{\ttfamily geometry\_option\_plasma} = 4: The plasma surface will be the last surface in the half radial grid of the \vmec~file specified by \parlink{wout\_filename}.
The poloidal angle used will be the straight-field-line coordinate, obtained by shifting the normal \vmec~poloidal angle by \vmec's $\lambda$ quantity.
This option exists in order to examine changes when using a different poloidal coordinate compared to {\ttfamily geometry\_option\_plasma} = 3.\\

{\ttfamily geometry\_option\_plasma} = 5: The plasma surface will be the flux surface with normalized poloidal flux
\parlink{efit\_psiN} taken from the {\ttfamily efit} file specified by \parlink{efit\_filename}.
}

\myhrule

\param{R0\_plasma}
{real}
{10.0}
{Only when \parlink{geometry\_option\_plasma} is 0 or 1.}
{Major radius of the plasma surface, when this surface is a plain circular torus.}

\myhrule

\param{a\_plasma}
{real}
{0.5}
{Only when \parlink{geometry\_option\_plasma} is 0 or 1.}
{Minor radius of the plasma surface, when this surface is a plain circular torus.}

\myhrule

\param{nfp\_imposed}
{integer}
{1}
{Only when \parlink{geometry\_option\_plasma} is 0 or 1.}
{When the plasma surface is a plain circular torus, only toroidal mode numbers that are a multiple of this parameter will be considered.
This parameter thus plays a role like \vmec's {\ttfamily nfp} (number of field periods),
and is used when {\ttfamily nfp} is not already loaded from a \vmec~file.}

\myhrule

\param{wout\_filename}
{string}
{`'}
{Only when \parlink{geometry\_option\_plasma} is 2, 3, or 4.}
{Name of the \vmec~{\ttfamily wout} output file which will be used for the plasma surface.
You can use either a \netCDF~or {\ttfamily ASCII} format file.}

\myhrule

\param{efit\_filename}
{string}
{`'}
{Only when \parlink{geometry\_option\_plasma} is 5.}
{Name of the {\ttfamily efit} output file which will be used for the plasma surface.}

\myhrule

\param{efit\_psiN}
{real}
{0.98}
{Only when \parlink{geometry\_option\_plasma} is 5.}
{Value of normalized poloidal flux at which to select a flux surface from the {\ttfamily efit} input file.
A value of 1 corresponds to the last closed flux surface, and 0 corresponds to the magnetic axis.}

\myhrule

\param{efit\_num\_modes}
{integer}
{10}
{Only when \parlink{geometry\_option\_plasma} is 5.}
{Controls the number of Fourier modes used to represent $R(\theta)$ and $Z(\theta)$ for the shape of
the plasma surface. Each of these functions will be expanded in a sum of functions $\sin(m\theta)$ and $\cos(m\theta)$,
where $m$ ranges from 0 to {\ttfamily efit\_num\_modes}$-1$.}

\section{Geometry parameters for the coil winding surface}

\param{geometry\_option\_coil}
{integer}
{0}
{Always}
{This option controls which type of geometry is used for the coil surface.\\

{\ttfamily geometry\_option\_coil} = 0: The coil surface will be a plain circular torus. The major radius will be the 
same as the plasma surface: either \parlink{R0\_plasma} if \parlink{geometry\_option\_plasma} is 0 or 1, or {\ttfamily Rmajor\_p} from the \vmec~{\ttfamily wout} file
if  \parlink{geometry\_option\_plasma} is 2.
     The minor radius will be \parlink{a\_coil}.\\

{\ttfamily geometry\_option\_coil} = 1: Identical to option 0, except the major radius of the coil surface will be set by \parlink{R0\_coil}.\\

{\ttfamily geometry\_option\_coil} = 2: The coil surface will computing by expanding the plasma surface uniformly by a distance \parlink{separation\_coil}.\\

{\ttfamily geometry\_option\_coil} = 3: The coil surface will be the `coil' surface in the \nescoil~`nescin' input file specified by \parlink{nescin\_filename\_coil}.
}

\myhrule

\param{R0\_coil}
{real}
{10.0}
{Only when \parlink{geometry\_option\_coil} is 1.}
{Major radius of the coil surface, when this surface is a plain circular torus.}

\myhrule

\param{a\_coil}
{real}
{1.0}
{Only when \parlink{geometry\_option\_coil} is 0 or 1.}
{Minor radius of the coil surface, when this surface is a plain circular torus.}


\myhrule

\param{separation\_coil}
{real}
{0.2}
{Only when \parlink{geometry\_option\_coil} is 2.}
{Amount by which the coil surface is offset from the plasma surface.}

\myhrule

\param{nescin\_filename\_coil}
{string}
{`'}
{Only when \parlink{geometry\_option\_coil} is 3.}
{Name of a \nescoil~{\ttfamily nescin} input file. The coil surface from
this file will be used as the coil surface for \regcoil.}


\section{Parameters related to the regularization weight}

\param{N\_lambdas}
{integer}
{4}
{Always}
{Number of values of $\lambda$ for which the problem is solved.}

\myhrule

\param{lambda\_max}
{real}
{10.0}
{Always}
{Maximum value of $\lambda$ for which the problem is solved.}

\myhrule

\param{lambda\_min}
{real}
{0.1}
{Always}
{Minimum nonzero value of $\lambda$ for which the problem is solved.
Note that the problem is always solved for $\lambda=0$ in addition to
the nonzero values.}

